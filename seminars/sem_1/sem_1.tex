\documentclass{article}
% =======PACKAGES=======
% FORMATTING
\usepackage[margin=0.625in]{geometry}
\usepackage{parskip, setspace}
\setstretch{1.15}
% TYPESETTING - MATH
\usepackage{amsmath, amsfonts}
\usepackage[ruled, linesnumbered, noend]{algorithm2e}
\usepackage{listings}
\usepackage{xcolor}

\lstdefinestyle{mystyle}{
    backgroundcolor=\color{lightgray},   
    commentstyle=\color{darkgray},
    keywordstyle=\color{red},
    numberstyle=\color{black},
    stringstyle=\color{violet},
    basicstyle=\ttfamily\footnotesize,
    breakatwhitespace=false,         
    breaklines=true,                 
    captionpos=b,                    
    keepspaces=true,                 
    numbers=left,                    
    numbersep=5pt,                  
    showspaces=false,                
    showstringspaces=false,
    showtabs=false,                  
    tabsize=2
}
\lstset{style=mystyle}
% RICH
\usepackage{graphicx, caption}
\usepackage{hyperref}
% BIBLIOGRAPHY
\usepackage[
backend=biber,
sorting=ynt
]{biblatex}
\addbibresource{sem_1.bib}

% =======TITLE=======
\title{\vspace*{-0.625in}CS 565: Scientific Computing \\ Seminar 1: A Survey of Psuedo-Random Number Generators}
\author{Nathan Chapman}
\date{\today}

\begin{document}

    \maketitle

    \section*{Introduction}

        Randomness is a crucial part of scientific computing due to its ability to give deterministic algorithms the ability to adjust in ways they aren't otherwise able.  A notable application of this is in computational optimization.  Though, a problem arises when trying to get a sense of randomness on a machine that is built to be deterministic.  The solution to this problem is in \emph{psuedo}-random number generators (PRNG).

        While computers were built to be deterministic, they still exhibit a source of randomness (i.e. entropy).  Using this source as a PRNG is possible, but too slow for any modern application where the programer needs many generations\cite{WhyNotRandomDevice}.  Because of its need, there have been many developments over the years for more efficient (in some cases \emph{much} more) PRNGs.  What follows is a survey of some of the most employed PRNGs used in scientific computing.
    
    \section*{Methods}

        The standard library of the numerical, scientific computing programming language \emph{Julia} contains two main algorithms to generate psuedo-random numbers\cite{Julia-2017}: \texttt{Xoshiro256++} and \texttt{MersenneTwister}.  As a historical note, \texttt{Lehmer} generation is also investigated.

        \subsection*{Lehmer}

            The Lhemer PRNG (also known as the Park–Miller random number generator) is one of the oldest PRNGs, dating back to the mid 1950s being created by VonNeumann, but it is a simple generator that can still lead to sufficient statistical properties.

            The Lhemer PRNG finds its roots as a linear congruential generator (LCG) taking the form

            \begin{equation*}
                X_{k + 1} = a X_k \mod m
            \end{equation*}

            where the modulus $m$ is a prime number or a power of a prime number, the multiplier $a$ is an element of high multiplicative order modulo $m$ (e.g., a primitive root modulo $n$), and the seed $X_0$ is coprime to $m$.  One of the most popular implementations is the MINSTD: $m = 2^{31} - 1,\ a = 48271$.

            The Lhemer PRNG shows up in several places including: \emph{Julia}'s \texttt{StableRNGs.jl}\cite{StableRNGs}, C++11's as \texttt{minstd\_rand}\cite{CppLhemer}, and the GNU Scientific Library\cite{GNUSciLib}.

            This method reached poputlarity due to several factors:
            
            \begin{itemize}
                \item its ability to only use 32-bit arithmetic using Schrage's method; thus being able to be implemented on GPU hardware
                \item It passes Big Crush in TestU01\cite{TestU01}
                \item It is fast and memory efficient (one modulo-m number, often 32 or 64 bits to retain state)
            \end{itemize}
\pagebreak
            Though there are some drawbacks:

            \begin{itemize}
                \item It fails the TMFn test from PractRand\cite{Practrand}
                \item A prime modulus implies modular reduction requires a double-width product and an explicit reduction step.
            \end{itemize}
        
        \subsection*{MersenneTwister}

            The Mersenne Twister PRNG is one of, if not \emph{the}, most popular and widely used PRNGs ever created.  This PRNG can be found as the default PRNG on many platforms, such as: Python\cite{Python}, Microsoft Excel\cite{Excel}, MATLAB\cite{MATLAB}, C++\cite{CppMT}, Mathematica\cite{Mathematica}, and CUDA\cite{CUDA}.

            The ubiquity of the Mersenne Twister could be due to several advantages it has over other PRNGs:

            \begin{itemize}
                \item It passes most of the standard statistical benchmarks\cite{TestU01}.
                \item It has a very long period; $2^{19937} - 1$ for the ``19937'' implementation used most.
                \item It has a GPU implmentation\cite{MTGP}.
            \end{itemize}

            While there are these advantages, Mersenne Twister based PRNGs still face some challenges when compared to some other methods.  These challenges include:

            \begin{itemize}
                \item A relatively large state buffer (the TinyMT variant addresses this problem\cite{TinyMT}).
                \item A generation speed that is relatively slow compared to more modern algorithms (the SFMT variant addresses this problem\cite{SFMT}).
                \item A linear complexity highlighted in both Crush and BigCrush\cite{TestU01}.
                \item An ability to predict all future generations after observing a certain number; e.g. for MT19937, you only need to generate 624 before being able to predict.  This predictablility makes MT19937 unsuitable for cryptographic applications.  The CryptMT variant addresses this problem\cite{CryptMT}.
            \end{itemize}

            The Mersenne Twister has several key features that define its mathematical behavior:

            \begin{itemize}
                \item It generates integers in the range $[0, 2^w - 1]$, for a $w$-bit word length.
                \item It is based on a matrix linear recurrence over a finite binary field $\mathbf{F}_2$.
                \item It is based on the generalised feedback shift register (GFSR)\cite{TwistedGFSR}.
            \end{itemize}

            The Mersenne Twister PRNG produces output based on two primary steps:

            \begin{enumerate}
                \item Generate a sequence $x_i$ of from a recurrence relation.
                \item Define a ``tempering'' matrix $T$ over the binary field $\mathbf{F}_2$
                \item Return results of the form $x_i^T$
            \end{enumerate}

        \subsection*{Xo(ro)shiro}

            While the Mersenne Twister algorithm has been quite sufficient for several years, recent improvements have come from a class of PRNGs known as the ``Xoshiro'' (xor-shift-rotate) and ``Xoroshiro'' (xor-rotate-shift-rotate).  These PRNGs come with several advantages over their older competitors\cite{XoroshiroPaper, Shootout}:

            \begin{itemize}
                \item Sub nanosecond generation
                \item A state space of only 256 bits
                \item If you only need 64-bit floats, xoshiro256+ is 15\% faster with the same properties
                \item A period of $2^{256} - 1$, and thus provides $2^{128}$ non-overlapping sequences of length $2^{128}$
                \item \texttt{Xoroshiro128} family is the same speed in half the space
                \item \texttt{Xoshiro128} family is 32 bits and thus applicable to be used on GPUs
                \item Every n/64-tuple of consecutive 64-bit values appears exactly once in the output, except for the zero tuple (and this is the largest possible dimension).
            \end{itemize}

            While there are several advantages to this class, there is still at least one drawback: the lowest bits have linear complexity.  Because of this, the isolation of these bits will fail the MatrixRank and LinearComp tests\cite{LowComp, TestU01}.  That being said, since only the lowest bits have linear complexity, the effects will be insignificant when considered in the context of the whole generator\cite{LowComp}.

            Because of these advantages, the Xo(ro)shiro class of PRNGs have been implented platforms such as javascript, rust, java, .net, erlang, FORTRAN, julia, lua, IoT (mbed and zephyr).

    \section*{Results}

        \begin{itemize}
            \item Show these algorithms in practice
            \item Quality\cite{Shootout}: Compare xoshiro and relatives using BigCrush suite of tests\cite{TestU01} and\cite{HammingWeightDependencies}
            \item Vectorization using AVX2 (Advanced Vector Extensions) vectorization\cite{Shootout}
        \end{itemize}

    \section*{Discussion}

        \begin{itemize}
            \item Compare and contrast each of these algorithms
        \end{itemize}

    \section*{Conclusion}
    
    \newpage
    \printbibliography

\end{document}