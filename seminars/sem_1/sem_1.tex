\documentclass{article}
% =======PACKAGES=======
% FORMATTING
\usepackage[margin=0.625in]{geometry}
\usepackage{parskip, setspace}
\setstretch{1.15}
\renewcommand{\arraystretch}{1.25}
% TYPESETTING - MATH
\usepackage{amsmath, amsfonts}
\usepackage[ruled, linesnumbered, noend]{algorithm2e}
\usepackage{listings}
\usepackage{xcolor}

\lstdefinestyle{mystyle}{
    backgroundcolor=\color{lightgray},   
    commentstyle=\color{darkgray},
    keywordstyle=\color{red},
    numberstyle=\color{black},
    stringstyle=\color{violet},
    basicstyle=\ttfamily\footnotesize,
    breakatwhitespace=false,         
    breaklines=true,                 
    captionpos=b,                    
    keepspaces=true,                 
    numbers=left,                    
    numbersep=5pt,                  
    showspaces=false,                
    showstringspaces=false,
    showtabs=false,                  
    tabsize=2
}
\lstset{style=mystyle}
% RICH
\usepackage{graphicx, caption}
\usepackage{hyperref}
% BIBLIOGRAPHY
\usepackage[
backend=biber,
sorting=ynt
]{biblatex}
\addbibresource{sem_1.bib}

% =======TITLE=======
\title{\vspace*{-0.625in}CS 565: Scientific Computing \\ Seminar 1: A Survey of Psuedo-Random Number Generators}
\author{Nathan Chapman}
\date{\today}

\begin{document}

    \maketitle

    \section*{Introduction}

        Randomness is a crucial part of scientific computing due to its ability to give deterministic algorithms the ability to adjust in ways they aren't otherwise able.  A notable application of this is in computational optimization.  Though, a problem arises when trying to get a sense of randomness on a machine that is built to be deterministic.  The solution to this problem is in \emph{psuedo}-random number generators (PRNG).

        While computers were built to be deterministic, they still exhibit a source of randomness (i.e. entropy).  Using this source as a PRNG is possible, but too slow for any modern application where the programer needs many generations\cite{WhyNotRandomDevice}.  Because of its need, there have been many developments over the years for more efficient (in some cases \emph{much} more) PRNGs.  What follows is a survey of some of the most employed PRNGs used in scientific computing.
    
    \section*{Methods}

        The standard library of the numerical, scientific computing programming language \emph{Julia} contains two main algorithms to generate psuedo-random numbers\cite{Julia-2017}: \texttt{Xoshiro256++} and \texttt{MersenneTwister}.  As a historical note, \texttt{Lehmer} generation is also investigated.

        \subsection*{Lehmer}

            The Lhemer PRNG (also known as the Park–Miller random number generator) is one of the oldest PRNGs, dating back to the mid 1950s being created by VonNeumann, but it is a simple generator that can still lead to sufficient statistical properties.

            The Lhemer PRNG finds its roots as a linear congruential generator (LCG) taking the form

            \begin{equation*}
                X_{k + 1} = a X_k \mod m
            \end{equation*}

            where the modulus $m$ is a prime number or a power of a prime number, the multiplier $a$ is an element of high multiplicative order modulo $m$ (e.g., a primitive root modulo $n$), and the seed $X_0$ is coprime to $m$.  One of the most popular implementations is the MINSTD: $m = 2^{31} - 1,\ a = 48271$.

            The Lhemer PRNG shows up in several places including: \emph{Julia}'s \texttt{StableRNGs.jl}\cite{StableRNGs}, C++11's as \texttt{minstd\_rand}\cite{CppLhemer}, and the GNU Scientific Library\cite{GNUSciLib}.

            This method reached poputlarity due to several factors:
            
            \begin{itemize}
                \item its ability to only use 32-bit arithmetic using Schrage's method; thus being able to be implemented on GPU hardware
                \item It passes Big Crush in TestU01\cite{TestU01}
                \item It is fast and memory efficient (one modulo-m number, often 32 or 64 bits to retain state)
            \end{itemize}
\pagebreak
            Though there are some drawbacks:

            \begin{itemize}
                \item It fails the TMFn test from PractRand\cite{Practrand}
                \item A prime modulus implies modular reduction requires a double-width product and an explicit reduction step.
            \end{itemize}
        
        \subsection*{MersenneTwister}

            The Mersenne Twister has several key features that define its mathematical behavior:

            \begin{itemize}
                \item It generates integers in the range $[0, 2^w - 1]$, for a $w$-bit word length.
                \item It is based on a matrix linear recurrence over a finite binary field $\mathbf{F}_2$.
                \item It is based on the generalised feedback shift register (GFSR)\cite{TwistedGFSR}.
            \end{itemize}

            The Mersenne Twister PRNG produces output based on three primary steps:

            \begin{enumerate}
                \item Generate a sequence $x_i$ of from a recurrence relation.
                \item Define a ``tempering'' matrix $T$ over the binary field $\mathbf{F}_2$
                \item Return results of the form $x_i^T$
            \end{enumerate}

        \subsection*{Xo(ro)shiro}

            While the Mersenne Twister algorithm has been quite sufficient for several years, recent improvements have come from a class of PRNGs known as the ``Xoshiro'' (xor-shift-rotate) and ``Xoroshiro'' (xor-rotate-shift-rotate).  One of the biggest advantages of using a Xo(ro)shiro class PRNG is its ability to use a \emph{jump polynomial}\cite{Shootout}:

            \begin{quote}
                \textit{All generators, being based on linear recurrences, provide jump functions that make it possible to simulate any number of calls to the next-state function in constant time, once a suitable jump polynomial has been computed. We provide ready-made jump functions for a number of calls equal to the square root of the period, to make it easy generating non-overlapping sequences for parallel computations, and equal to the cube of the fourth root of the period, to make it possible to generate independent sequences on different parallel processors.}
            \end{quote}

    \section*{Results}

        While these are of great interest in a theoretical sense, we also must be diligent in considering their real-world implemnetation. Table \ref{tbl:comparisons} compares the step period, the time period per 64 bits, and the number of cycles per byte\cite{Shootout}.

        \begin{table}[h]
        \begin{center}
        \begin{tabular}{c|c|c|c|c}
            PRNG         & Footprint (bits) & Period          & ns/64 bits & cycles/B \\
            \hline
            xoshiro256+  & 256              & $2^{256} - 1$   & 0.61       & 0.27 \\
            \hline
            xoshiro256++ & 256              & $2^{256} - 1$   & 0.75       & 0.34 \\
            \hline
            xoshiro256** & 256              & $2^{256} - 1$   & 0.75       & 0.34 \\
            \hline
            MT19937-64   & 20032            & $2^{19937} - 1$ & 1.36       & 0.62 \\
            \hline
            SFMT19937    & 20032            & $2^{19937} - 1$ & 0.93       & 0.42 \\
            \hline
            TinyMT-64    & 256              & $2^{127} - 1$   & 2.76       & 1.25
        \end{tabular}
        \end{center}
            \caption{\label{tbl:comparisons}Comparison of PRNG properties}
        \end{table}

        These algorithms have been tested and compared with the standard Big Crush suite\cite{TestU01} and a custom test for Hamming-weight dependencies\cite{HammingWeightDependencies}.  Additionally, we consider the footprint of each algorithm to determine its ``bang for the buck'' defined as\cite{Shootout}
        \begin{quote}
            \textit{a generator gives \textbf{``bang-for-the-buck''} if the base-2 logarithm of the period is close to the footprint.}
        \end{quote}

    \section*{Discussion}

        As the Lhemer PRNG, has been included here merely as historical interest, the interesting competition is thus only between the MersenneTwister and Xo(ro)shiro families of PRNGs.  Each of these competitive methods has been implemented on various platforms, some originally defaulting to Mersenne Twister and recently moved to Xoroshiro.

        The \emph{Mersenne Twister} PRNG can be found as the default on many platforms, most notably of which are: Python\cite{Python}, Microsoft Excel\cite{Excel}, MATLAB\cite{MATLAB}, C++\cite{CppMT}, Mathematica\cite{Mathematica}, and CUDA\cite{CUDA}.  Likewise, due to its increases in popularity, the \emph{Xo(ro)shiro} class of PRNGs have been implented platforms such as javascript, rust, java, .net, erlang, FORTRAN, julia, lua, IoT (mbed and zephyr).

        While both methods have been implemented in many cases as the default PRNG, each has its own set of advantages. The highlights of how Mersenne Twister leads the charge are:

        \begin{itemize}
            \item It passes most of the standard statistical benchmarks\cite{TestU01}.
            \item It has a very long period; $2^{19937} - 1$ for the ``19937'' implementation used most.
            \item It has a GPU implmentation\cite{MTGP}.
        \end{itemize}

        Likewise, the Xoshiro family also comes with several advantages over its older competitors\cite{XoroshiroPaper, Shootout}:

        \begin{itemize}
            \item Sub nanosecond generation
            \item A state space of only 256 bits
            \item If you only need 64-bit floats, xoshiro256+ is 15\% faster with the same properties
            \item A period of $2^{256} - 1$, and thus provides $2^{128}$ non-overlapping sequences of length $2^{128}$
            \item \texttt{Xoroshiro128} family is the same speed in half the space
            \item \texttt{Xoshiro128} family is 32 bits and thus applicable to be used on GPUs
            \item Every $n/64$-tuple of consecutive 64-bit values appears exactly once in the output, except for the zero tuple (and this is the largest possible dimension).
        \end{itemize}

        While there are these advantages, Mersenne Twister based PRNGs still face some challenges when compared to some other methods.  These challenges include:

        \begin{itemize}
            \item A relatively large state buffer (the TinyMT variant addresses this problem\cite{TinyMT}).
            \item A generation speed that is relatively slow compared to more modern algorithms (the SFMT variant addresses this problem\cite{SFMT}).
            \item A linear complexity highlighted in both Crush and BigCrush\cite{TestU01}.
            \item An ability to predict all future generations after observing a certain number; e.g. for MT19937, you only need to generate 624 before being able to predict.  This predictablility makes MT19937 unsuitable for cryptographic applications.  The CryptMT variant addresses this problem\cite{CryptMT}.
        \end{itemize}

        Much like Mersenne Twister, there is still at least one challenge the Xoshiro family faces: the lowest bits have linear complexity.  Because of this, the isolation of these bits will fail the MatrixRank and LinearComp tests\cite{LowComp, TestU01}.  That being said, since only the lowest bits have linear complexity, the effects will be insignificant when considered in the context of the whole generator\cite{LowComp}.

\pagebreak

    \section*{Conclusion}

        Due to its importance in scientiic computing and other applications, many psuedo-random number generators have been created and polished over the years.  Modern methods have moved away from ``mathematical'' generation, such as modular recurrences towards a more ``computational'' method, incorporating concepts such as recurring bit shifts.  These modern methods, such as the Mersenne Twister and Xo(ro)shiro classes of PRNGs, are able to provide incredible improvements in both speed of generation and quality, even for different hardware.  
        
        While there have been numerous varitations to Mersenne wister based algorithms to mitigate problems such as large state space requiring large amounts of memory, and speed of generation, Xo(ro)shiro based methods have been created that perform better than all their Mersenne counter-parts in every considered metric.  Because of this overall advantage, many substantial platforms such as  Javascript, Rust, Java, and Julia have have decided to make Xo(ro)shiro their default moving forward.
    
    \printbibliography

\end{document}