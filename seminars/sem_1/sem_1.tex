\documentclass{article}
% =======PACKAGES=======
% FORMATTING
\usepackage[margin=0.625in]{geometry}
\usepackage{parskip, setspace}
\setstretch{1.15}
% TYPESETTING - MATH
\usepackage{amsmath, amsfonts}
\usepackage[ruled, linesnumbered, noend]{algorithm2e}
\usepackage{listings}
\usepackage{xcolor}

\lstdefinestyle{mystyle}{
    backgroundcolor=\color{lightgray},   
    commentstyle=\color{darkgray},
    keywordstyle=\color{red},
    numberstyle=\color{black},
    stringstyle=\color{violet},
    basicstyle=\ttfamily\footnotesize,
    breakatwhitespace=false,         
    breaklines=true,                 
    captionpos=b,                    
    keepspaces=true,                 
    numbers=left,                    
    numbersep=5pt,                  
    showspaces=false,                
    showstringspaces=false,
    showtabs=false,                  
    tabsize=2
}
\lstset{style=mystyle}
% RICH
\usepackage{graphicx, caption}
\usepackage{hyperref}
% BIBLIOGRAPHY
\usepackage[
backend=biber,
sorting=ynt
]{biblatex}
\addbibresource{bib.bib}

% =======TITLE=======
\title{\vspace*{-0.625in}CS 565: Scientific Computing \\ Seminar 1: Psuedo-Random Number Generators}
\author{Nathan Chapman}
\date{\today}

\begin{document}

    \maketitle

    \section*{Introduction}
    
    \section*{Methods}

        \begin{itemize}
            \item TaskLocalRNG: a token that represents use of the currently active Task-local stream, deterministically seeded from the parent task, or by RandomDevice (with system randomness) at program start
            \item Xoshiro: generates a high-quality stream of random numbers with a small state vector and high performance using the Xoshiro256++ algorithm
            \item RandomDevice: for OS-provided entropy. This may be used for cryptographically secure random numbers (CS(P)RNG).
            \item MersenneTwister: an alternate high-quality PRNG which was the default in older versions of Julia, and is also quite fast, but requires much more space to store the state vector and generate a random sequence.
            \item Parallel versions
            \item Sample from normal or exponential distribution
        \end{itemize}

    \section*{Results}

    \section*{Discussion}

    \section*{Conclusion}

    \printbibliography

\end{document}