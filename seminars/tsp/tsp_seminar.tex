\documentclass{article}
% =======PACKAGES=======
% FORMATTING
\usepackage[margin=0.625in]{geometry}
\usepackage{parskip, setspace}
\setstretch{1.15}
\renewcommand{\arraystretch}{1.25}
% TYPESETTING - MATH
\usepackage{amsmath, amsfonts}
\usepackage[ruled, linesnumbered, noend]{algorithm2e}
% \usepackage{listings}
% \usepackage{xcolor}

% \lstdefinestyle{mystyle}{
%     backgroundcolor=\color{lightgray},   
%     commentstyle=\color{darkgray},
%     keywordstyle=\color{red},
%     numberstyle=\color{black},
%     stringstyle=\color{violet},
%     basicstyle=\ttfamily\footnotesize,
%     breakatwhitespace=false,         
%     breaklines=true,                 
%     captionpos=b,                    
%     keepspaces=true,                 
%     numbers=left,                    
%     numbersep=5pt,                  
%     showspaces=false,                
%     showstringspaces=false,
%     showtabs=false,                  
%     tabsize=2
% }
% \lstset{style=mystyle}
% RICH
\usepackage{graphicx, caption}
\usepackage{hyperref}
% BIBLIOGRAPHY
\usepackage[
backend=biber,
sorting=ynt
]{biblatex}
\addbibresource{bib.bib}

% =======TITLE=======
\title{\vspace*{-0.625in}CS 565: Scientific Computing \\ Seminar 2: Methods of Combinatorial Optimization and the TSP}
\author{Nathan Chapman}
\date{\today}

\begin{document}

    \maketitle

    \section*{Introduction}

    \section*{Methods}

        \subsection*{Gravitational Search Algorithm}

            \begin{itemize}
                \item Based on N-Body gravitational simulation
                \item Just use R instead of $R^2$
                \item Net force acting on an object is a randomly weighted linear combination of all the other forces.
            \end{itemize}

        \subsection*{Branch \& Cut}

    \section*{Results}

        Show results of each method on both the TSP and ATSP.

    \section*{Discussion}

        Compare and contrast the different methods with their pros and cons

    \section*{Conclusion}

\end{document}
