\documentclass{article}
% =======PACKAGES=======
% FORMATTING
\usepackage[margin=0.625in]{geometry}
\usepackage{parskip, setspace}
\setstretch{1.15}
\renewcommand{\arraystretch}{1.25}
% TYPESETTING - MATH
\usepackage{amsmath, amsfonts}
\usepackage[ruled, linesnumbered, noend]{algorithm2e}
% RICH
\usepackage{graphicx, caption}
\usepackage{hyperref}
% BIBLIOGRAPHY
\usepackage[
backend=biber,
sorting=ynt
]{biblatex}
\addbibresource{bib.bib}

% =======TITLE=======
\title{\vspace*{-0.625in}CS 565: Scientific Computing \\ Project 1: Random Search \& Empirical Gradient Descent}
\author{Nathan Chapman and Andrew Struthers}
\date{\today}

\begin{document}

    \maketitle

    \section{Introduction}

    \section{Methods}

        \subsection{Language}

            \subsubsection{Julia}

            \subsubsection{C++}

            \subsubsection{CUDA}

        \subsection{Pseudo-random Number Generation}

            \subsubsection{Xoshiro256++}

            \subsubsection{Mersenne Twister}

        \subsection{Benchmarks}

        \subsection{Empirical Gradient Descent}

    \section{Results}

    \section{Discussion}
        Compare empirical gradient to other standard methods in numerical optimization of geometric structures:
        \begin{itemize}
            \item Barzilai-Borwein method
            \item Wolfe conditions and Line search
            \item Conjugate gradient method
            \item Proximal gradient method
            \item Stochastic gradient descent
            \item Mirror descent
            \item Broyden–Fletcher–Goldfarb–Shanno algorithm
            \item Davidon–Fletcher–Powell formula
            \item Nelder–Mead method
            \item Gauss–Newton algorithm
            \item Quantum annealing
            \item Hill climbing
        \end{itemize}

    \section{Conclusion}

\end{document}
