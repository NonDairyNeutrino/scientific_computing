\documentclass{article}
% =======PACKAGES=======
% FORMATTING
\usepackage[margin=0.625in]{geometry}
\usepackage{parskip, setspace}
\setstretch{1.15}
\renewcommand{\arraystretch}{1.25}
% TYPESETTING - MATH
\usepackage{amsmath, amsfonts}
\usepackage[ruled, linesnumbered, noend]{algorithm2e}
% RICH
\usepackage{graphicx, caption}
\usepackage{hyperref}
% BIBLIOGRAPHY
\usepackage{natbib}
\bibliographystyle{unsrt}

% =======TITLE=======
\title{\vspace*{-0.625in}CS 565: Scientific Computing \\ Project 1: Random Search \& Empirical Gradient Descent\vspace*{-0.25in}}
\author{Nathan Chapman and Andrew Struthers}
\date{\today}

\begin{document}

    \maketitle

\tableofcontents
\pagebreak
    \section{Introduction}

        Optimization algorithms play a pivotal role in various fields, ranging from machine learning and artificial intelligence to engineering and finance. Mathematical optimization is often full of high-dimension, intractable problems. These problems have solution spaces that are very difficult, if not impossible, to visualize and work with using standard approaches.  Mathematical optimization problems also typically suffer from complex geometric solution spaces, full of plateaus, holes, and other features that make effective traversal difficult. Efficient and reliable algorithms have been created that navigate complex landscapes of these high-dimensional benchmark problem. 

        Classic benchmark problems, such as Ackley's, De Jong's, and Schwefel's functions, test various strengths and weaknesses of these algorithms. The classic benchmark functions contain many complex geometric structures in their solution spaces, making them effective at rigorously testing new optimization algorithms. With the ever-increasing complexity of real-world optimization challenges, the development of consistent, effective, and fast optimization techniques has become a popular area of study among Computer Scientists. 

        This paper discusses and analyzes the intersection of pseudo-random number generators (PRNGs) and their role in generating solution vectors for these multi-dimensional benchmark problems. In addition to investigating the impact of various pseudo-random number generators (PRNGs) on initial solution generation, we also use empirical gradient descent to improve upon the initial solution population. By using statistically strong PRNGs to generate initial solutions, then applying a heuristic-based empirical gradient descent, we can further explore the solution space with the aim of seeking the global optimum.

        Optimization algorithms, such as empirical gradient descent, have proven successful in mathematical optimization problems; however, their performance relies heavily on the initial solution vectors. By combining strong PRNGs with empirical gradient descent, we hope to traverse complex fitness landscapes posed by benchmark problems effectively and reliably.

        The benchmark problems used in this paper each have distinct characteristics that rigorously test optimization algorithms. Some of the functions, like Ackley’s, Egg Holder, and Rastrigin, are known for their non-convexity and multiple local optima. Functions like Rosenbrock’s Saddle contain valleys and steep ridges, while functions like Stretch V Sine Wave pose varying solution spaces and are computationally expensive to compute.

        We use a diverse set of generators, where each generator has unique properties that might impact optimization outcomes. Because we are combining a random based initial solution approach with empirical gradient descent, the choice of PRNG will influence not only the diversity of the initial solution pool the trajectories followed during gradient descent iterations. The Mersenne Twister is a classic generator used very frequently in many applications due to its long period and statistical properties. We will also be using Xoshiro256++, which is hardware based and is capable of fast execution without sacrificing excellent statistical performance. We also consider the Lagged Fibonacci Generator (LFib), a simple but powerful recurrence based PRNG. Additionally, we look at using CUDA for massively parallel experimentation with a larger number of experiments for each algorithm. CUDA has native implementations of the XORWow, Mersenne Twister, and other parallelizable PRNG algorithms.

        Through this study, we seek to devise a strategy of combining PRNGs and empirical gradient descent to navigate complex search spaces efficiently. This combined approach will improve the optimization algorithm convergence as well as the ability to find the global optima. This combined approach aims utilizes the best parts of both randomness and deterministic refinement in solving high-dimensional optimization problems.

\pagebreak
    \section{Methods}

        \subsection{Pseudo-random Number Generation}

            \subsubsection{Xoshiro256++}
Xoshiro256++ is fast and possesses high-quality statistics when generating random numbers. This is a generator that belongs to the family of XORShift generators. This generator uses hardware-based bit operations, as is standard in the XORShift family. It is designed to produce long and statistically valid sequences of numbers without the overhead required for recursion-based generators. The ``256" in its name means that there is a 256-bit internal state. Xoshiro256++ is part of the XORShift family, but it extends upon the speed and statistical properties found in simpler implementations. Because of the 256-bit internal state, it has a period of $2^{256}$, making it a longer period than most other XORShift like generators. 

There is a predecessor of this generator, the Xoshiro256*, but this generator is stronger due to its combination of rotations and bitwise xorshift operations. The rotations as well as bitwise xor allows a much more random generation pattern compared to Xoshiro256*. Due to the hardware support for bitwise operations, this generator is incredibly fast. To get the most out of this generator, seeding the generator with a highly random initial state is necessary to avoid repetition and prediction.


            \subsubsection{Mersenne Twister} 
The Mersenne Twister is a very commonly used general purpose PRNG that introduces a long period, which was uncommon in other PRNGs around the time, which often suffered from having a short period. The Mersenne Twister algorithm has a period of $2^19937-1$. This makes the Mersenne Twister a very effective as a PRNG. Mersenne Twister is one of two PRNGs used in the IBM Statistical Package for the Social Sciences (SPSS) and is implemented in many different programming languages as a native or STL PRNG. While a long period doesn't necessarily equate to a high quality PRNG, the Mersenne Twister is also $k$-distributed to a 32-bit accuracy for every $1\leq k\leq623$. This algorithm is very fast, but it does require a state buffer to operate. The state buffer for the MT algorithm is quite large compared to other variants, so while it is quite fast and high quality, it does require more space to operate. On machines of sufficient size, this typically isn't a concern, but on smaller machines where memory is more limited, this could be unacceptable. One other drawback is that the MT can take a long time to generate output that passes various statistical randomness tests if the seed value isn't sufficiently random itself, so using operations like offsetting the start of the sequence can be beneficial. The Mersenne Twister is a very good standard PRNG, and as such, is implemented in many programming languages and software packages. We used the standard Mersenne Twister algorithm in both the Julia implementation and the CUDA implementation.

            \subsubsection{XORWow}
The XORWow algorithm is one of the standard PRNGs implemented in the CUDA library. It is part of the XORShift family of PRNGs and uses hardware level operations to achieve fast and efficient random number generation. CUDA uses XORWow as a standard implementation because it can be implemented to generate numbers in parallel, making full use of the GPU acceleration. XORWow repeatedly applies a bitwise XOR operation and a shift operation to its internal state, very similar to the basic XORShift, or more complex Xoshiro256, to produce a sequence of random numbers. Because of the parallel nature of this algorithm, different GPU threads can generate streams of random numbers independently, which allows an efficient utilization of the GPU when generating many random numbers. To ensure independence when generating numbers in parallel, each thread is initialized with a unique seed that relies on some combination of the block and thread IDs. This independence allows XORWow to produce random numbers with good statistical properties very quickly. Because of its many benefits, we used XORWow as one of the PRNGs in the CUDA implementation.

            \subsubsection{Philox}
The Philox PRNG is another standard CUDA implemented cryptographic-grade PRNG capable of parallel processing and generation. The Philox algorithm works by using combinations of multiplication and bitwise operations which are well supported at the hardware level, and as such can generate random numbers concurrently across multiple threads. Like in XORWow, CUDA creates unique seeds from a combination of block and thread IDs to guarantee independence. Philox, as a cryptographic-grade PRNG, has excellent statistical properties alongside long periods and good uniformity. Philox can generate both 32-bit and 64-bit random numbers if desired, which allows us to generate numbers with more precision if necessary. A cryptographic-grade PRNG is generally considered unnecessary compared to other more common implementations, but when doing optimization with thousands or millions of input vectors, making sure that the quality, unpredictability, and lack of repetition is strong becomes crucial. With CUDA, we can effectively parallelize the generation of these numbers, negating a lot of the downside to using a more complex cryptographic PRNG. 

In CUDA specifically, the standard implementation is the ``Philox(4, 32, 10)", which specifies some parameters used in the generator. The $32$ is representative of generating 32-bit numbers. Even though we could use 64-bit precision, the standard implementation is more than sufficient. The $4$ represents the amount of numbers Philox generates in each call, and the $10$ denotes the number of rounds in the Philox algorithm, where a round refers to the number of iterations Philox applies to each set of $4$ random numbers. A round consists of some mathematical operations, typically involving bitwise operations like the XORShift family, and modular arithmetic like recurrence based PRNGs. These operations are applied to the internal state of the PRNG, meaning that the Philox algorithm is applied 10 times to generate each set of $4$ 32-bit random numbers. This increases the overhead of the algorithm by a lot, but the tradeoff of computational efficiency is quality of the generated random numbers.


        \subsection{Benchmarks}
The benchmark functions used in this report are as follows:
\begin{enumerate}
\item Ackley's One:
    \begin{equation} \label{eq:ackley_one} \tag{Ackley's One}
    \begin{aligned}
        f(x) &= \sum_{i=1}^{n-1}\frac{1}{e^{0.2}}\sqrt{x_i^2+x_{i+1}^2}+3\left(cos\left(2x_i\right)+sin\left(2x_{i+1}\right)\right)
    \end{aligned}
    \end{equation}

\item Ackley's Two:
    \begin{equation} \label{eq:ackley_two} \tag{Ackley's Two}
    \begin{aligned}
        f(x) &= \sum_{i=1}^{n-1}20+e-\frac{20}{e^{0.2\sqrt{\frac{x_i^2+x_{i+1}^2}{2}}}}-e^{0.5\left(cos\left(2\pi\cdot x_i\right) + cos\left(2\pi\cdot x_{i+1}\right)\right)}
    \end{aligned}
    \end{equation}

\item 1st De Jong's:
    \begin{equation} \label{eq:de_jong} \tag{1st De Jong's}
    \begin{aligned}
        f(x) &= \sum_{i=1}^{n}x_i^2
    \end{aligned}
    \end{equation}

\item Egg Holder:
    \begin{equation} \label{eq:egg} \tag{Egg Holder}
    \begin{aligned}
        f(x) &= \sum_{i=1}^{n-1}-x_i\cdot sin\left(\sqrt{\left|x_i-x_{i+1}-47\right|}\right) - \left(x_{i+1}+47\right)\cdot sin\left(\sqrt{\left|x_{i+1}+47+\frac{x_i}{2}\right|}\right)
    \end{aligned}
    \end{equation}

\item Griewangk:
    \begin{equation} \label{eq:griewangk} \tag{Griewangk}
    \begin{aligned}
        f(x) &= \sum_{i=1}^{n}\frac{x_i^2}{4000}-\prod_{i=1}^{n}cos\left(\frac{x_i}{\sqrt{i}}\right)
    \end{aligned}
    \end{equation}

\item Rastrigin:
    \begin{equation} \label{eq:rastrigin} \tag{Rastrigin}
    \begin{aligned}
        f(x) &= 10\cdot n\cdot \sum_{i=1}^{n}\left(x_i^2-10\cdot cos\left(2\pi\cdot x_i\right)\right)
    \end{aligned}
    \end{equation}

\item Rosenbrock's Saddle:
    \begin{equation} \label{eq:rosenbrock} \tag{Rosenbrock's Saddle}
    \begin{aligned}
        f(x) &= \sum_{i=1}^{n-1}100\cdot\left(x_i^2-x_{i+1}\right)^2+\left(1-x_i\right)^2
    \end{aligned}
    \end{equation}

\item Schwefel:
    \begin{equation} \label{eq:schwefel} \tag{Schwefel}
    \begin{aligned}
        f(x) &= \left(418.9829\cdot n\right)-\sum_{i=1}^{n}-x_i\cdot sin\left(\sqrt{\left|x_i\right|}\right)
    \end{aligned}
    \end{equation}

\item Sine Envelope Sine Wave:
    \begin{equation} \label{eq:sesw} \tag{Sine Envelope Sine Wave}
    \begin{aligned}
        f(x) &= -\sum_{i=1}^{n-1}0.5+\frac{sin\left(x_i^2+x_{i+1}^2-0.5\right)^2}{\left(1+0.001\left(x_i^2+x_{i+1}^2\right)\right)^2}
    \end{aligned}
    \end{equation}

\item Stretched V Sine Wave:
    \begin{equation} \label{eq:svsw} \tag{Stretched V Sine Wave}
    \begin{aligned}
        f(x) &= \sum_{i=1}^{n-1}\left(\sqrt[4]{x_i^2+x_{i+1}^2}\cdot sin\left(50\sqrt[10]{x_i^2+x_{i+1}^2}\right)^2+1\right)
    \end{aligned}
    \end{equation}
\end{enumerate}
Each of these benchmark functions pose different fitness landscapes. The \ref{eq:de_jong}, for example, is very convex, with one global minimum and no local minium. Other equations, such as \ref{eq:ackley_one}, \ref{eq:ackley_two}, and \ref{eq:schwefel} have very intense landscapes with many local minima in the form of $n$-dimensional holes, where finding the global optimum requires an algorithm capable of traversing these wildly varied landscapes. Functions such as \ref{eq:sesw} and \ref{eq:svsw} have many local minima in a repeating fashion, with undulating walls formed by the various trigonometric functions. Traversing these diverse solution landscapes with an efficient algorithm is critical to finding the global optimum for each of these functions. Additionally, these functions are very complex and time intensive to compute, especially the functions that make heavy use of trigonometric functions. These functions are geometrically varied and computationally expensive to compute, making traversal very difficult.
        \subsection{Empirical Gradient Descent}

            Gradient descent(eq. \ref{eq:gradient}, \ref{eq:descent}, \ref{eq:while_gradient})
            
            \begin{align}
                \label{eq:gradient} \nabla f(x) &= \left\{ \frac{\partial f}{\partial x_1}, \frac{\partial f}{\partial x_2}, \ldots\right\} \\
                \label{eq:descent} x &\leftarrow x - \alpha \nabla f(x) \\
                \label{eq:while_gradient} \text{\texttt{while} } \Omega &< ||\nabla f(x)||
            \end{align}
            
            is a classical method in mathematical optimization because of its roots in calculus and intuitivity.  Gradient decsent has drawbacks when it comes to differentiability,  convexity, and curvature.  For example, the descent (\ref{eq:descent}) is updated as long as the size of the gradient is bigger than (eq \ref{eq:while_gradient}) zero ($\Omega = 0$) (or threshold when being done computationally $0 < \Omega \ll 1$).  As a first step to approximate the useful features of the gradient and mitigate these problems, the \emph{empirical} gradient varition can be used.  The key difference between these methods, is that instead of focusing on the size of the gradient vector (i.e. calculating its norm at every step until its close enough to zero) (eq. \ref{eq:while_gradient}), we focus solely on the change of the image of the function as (eq. \ref{eq:empirical_while})

            \begin{align}
                \label{eq:empirical_gradient} \nabla_e f(x) &= \left\{f\left(x + \delta \hat{k}_i\right) - f\left(x\right)\right\}_i \\
                \label{empirical_descent} x_{n} &\leftarrow x_{n-1} - \delta \nabla_e f(x_{n-1}) \\
                \label{eq:empirical_while} \text{\texttt{while} } f(x_{n}) &< f(x_{n-1})
            \end{align}

            where $\hat{k}_i$ is the unit vector in the $i$-th dimension.

            Because of this change in perspective, no vector norms need to be calculated, possibly leading to significant decreases in runtime.

\pagebreak
    \section{Results}

        \subsection{Random Search}

            \begin{table}[h]
            \begin{centering}
                \begin{tabular}{|c||c|c|c|c|c|c|}
                    \hline
                                & Average & Standard Deviation & Minimum & Maximum & Median & Time [s] \\
                    \hline
                    \hline
                    Ackley's One & 587.19 & 60.62 & 424.5 & 705.34 & 588.17 & 0.438 \\
                    \hline
                    Ackley's Two & 652.36 & 14.56 & 602.65 & 677.75 & 653.94 & 0.053 \\
                    \hline
                    1st De Jong's & 104726.64 & 14731.69 & 79401.66 & 136052.36 & 105329.46 & 0.022 \\
                    \hline
                    Egg Holder & -67.51 & 1564.88 & -3543.09 & 3168.68 & -199.16 & 0.119 \\
                    \hline
                    Griewangk & 635.73 & 129.05 & 354.25 & 841.71 & 663.16 & 0.065 \\
                    \hline
                    Rastrigin & $2.6 \times 10^{6}$ & 511998.81 & $1.6 \times 10^6$ & $2.9 \times 10^6$ & $2.5 \times 10^6$ & 0.025 \\
                    \hline
                    Rosenbrock & $5.5 \times 10^{10}$ & $1.5 \times 10^{10}$ & $2.6 \times 10^{10}$ & $8.2 \times 10^{10}$ & $5.6 \times 10^{10}$ & 0.055 \\
                    \hline
                    Schwefel's & 12265.91 & 898.52 & 10669.58 & 15107.91 & 12154.59 & 0.044 \\
                    \hline
                    SESW & -20.89 & 1.31 & -23.97 & -18.68 & -20.59 & 0.083 \\
                    \hline
                    SVSW & 94.07 & 7.9 & 77.61 & 122.17 & 93.86 & 0.081 \\
                    \hline
                \end{tabular}
                \caption{Julia with Xoshiro256++}
            \end{centering}
            \end{table}

        \begin{table}[h]
            \begin{centering}
                \begin{tabular}{|c||c|c|c|c|c|c|}
                    \hline
                                & Average & Standard Deviation & Minimum & Maximum & Median & Time [s] \\
                    \hline
                    \hline
                    Ackley One & 583.12 & 58.6578 & 304.542 & 842.013 & 583.285 & 0.00845 \\
                    \hline
                    Ackley Two & 583.037 & 16.5308 & 462.064 & 627.577 & 585.359 & 0.03711 \\
                    \hline
                    De Jong 1 & 100140 & 16334.8 & 29912.4 & 176241 & 99688.7 & 0.00042163\\
                    \hline
                    Egg Holder & -106.642 & 1565.33 & -8759.94 & 7308.14 & -99.2367 & 0.00992 \\
                    \hline
                    Griewangk & 625.04 & 102.136 & 199.357 & 1125.12 & 622.969 & 0.00793 \\
                    \hline
                    Rastrigin & 2.69884e+06 & 441123 & 902980 & 5.22212e+06 & 2.69134e+06 & 0.00374 \\
                    \hline
                    Rosenbrock's & 5.79781e+10 & 1.43681e+10 & 9.46702e+09 & 1.36466e+11 & 5.73733e+10 & 0.02876\\
                    \hline
                    Schwefel & 12556.5 & 1069.19 & 6795.11 & 17630.6 & 12568.6 & 0.00472 \\
                    \hline
                    SESW & -21.1838 & 1.25719 & -28.2668 & -16.7156 & -21.1149 & 0.03290 \\
                    \hline
                    SVSW & 303499 & 11836.1 & 225942 & 336821 & 303329 & 0.04492 \\
                    \hline
                \end{tabular}
                \caption{1000000 Experiments with CUDA using XORWow}
            \end{centering}
            \end{table}

        \begin{table}[h]
            \begin{centering}
                \begin{tabular}{|c||c|c|c|c|c|c|}
                    \hline
                                & Average & Standard Deviation & Minimum & Maximum & Median & Time [s] \\
                    \hline
                    \hline
                    Ackley One & 582.921 & 58.6353 & 308.961 & 843.791 & 583.149 & 0.00845 \\
                    \hline
                    Ackley Two & 583.059 & 16.5201 & 459.279 & 628.006 & 585.38 & 0.03711 \\
                    \hline
                    De Jong 1 & 100127 & 16334.8 & 30795.6 & 175801 & 99677.5 & 0.00042323\\
                    \hline
                    Egg Holder & -107.924 & 1563.63 & -7848.96 & 7437.9 & -102.868 & 0.00992 \\
                    \hline
                    Griewangk & 625.142 & 101.975 & 199.423 & 1121.05 & 623.185 & 0.00793 \\
                    \hline
                    Rastrigin & 2.69884e+06 & 441123 & 902980 & 5.22212e+06 & 2.69134e+06 & 0.00374 \\
                    \hline
                    Rosenbrock's & 5.79839e+10 & 1.43691e+10 & 9.19142e+09 & 1.34815e+11 & 5.73814e+10 & 0.02876 \\
                    \hline
                    Schwefel & 12559.7 & 1067.37 & 7388.06 & 17928.6 & 12570.3 & 0.00472\\
                    \hline
                    SESW & -21.1796 & 1.25551 & -28.1444 & -16.5571 & -21.1133 & 0.03290 \\
                    \hline
                    SVSW & 303499 & 11836.1 & 225942 & 336821 & 303329 & 0.04492 \\
                    \hline
                \end{tabular}
                \caption{1000000 Experiments with CUDA using Mersenne Twister}
            \end{centering}
            \end{table}

        \begin{table}[h]
            \begin{centering}
                \begin{tabular}{|c||c|c|c|c|c|c|}
                    \hline
                                & Average & Standard Deviation & Minimum & Maximum & Median & Time [s] \\
                    \hline
                    \hline
                    Ackley One & 583.075 & 58.607 & 304.909 & 829.551 & 583.199 & 0.00845 \\
                    \hline
                    Ackley Two & 583.097 & 16.5323 & 471.824 & 628.922 & 585.396 & 0.03711 \\
                    \hline
                    De Jong 1 & 100149 & 16344.2 & 32357.6 & 182461 & 99709 & 0.00042230\\
                    \hline
                    Egg Holder & -106.53 & 1565.91 & -8484.27 & 7582.61 & -99.0725 & 0.00992 \\
                    \hline
                    Griewangk & 625.134 & 102.023 & 205.271 & 1110.3 & 623.039 & 0.00793 \\
                    \hline
                    Rastrigin & 2.69884e+06 & 441123 & 902980 & 5.22212e+06 & 2.69134e+06 & 0.00374 \\
                    \hline
                    Rosenbrock's & 5.80077e+10 & 1.43466e+10 & 8.38982e+09 & 1.37005e+11 & 5.74165e+10 & 0.02876 \\
                    \hline
                    Schwefel & 12559.4 & 1068.21 & 7481.93 & 17714.3 & 12569.9 & 0.00472 \\
                    \hline
                    SESW & -21.1838 & 1.25621 & -27.9692 & -16.6082 & -21.115 & 0.03290\\
                    \hline
                    SVSW & 303516 & 11805.7 & 219669 & 336788 & 303327 & 0.04493\\
                    \hline
                \end{tabular}
                \caption{1000000 Experiments with CUDA using Philox}
            \end{centering}
            \end{table}

        \begin{table}[h]
            \begin{centering}
                \begin{tabular}{|c||c|c|c||c|c|c||c|c|c|}
                    \hline
                                & Average & Minimum & Time [s] & Average & Minimum & Time [s] & Average & Minimum & Time [s] \\
                    \hline
                    \hline
                    Ackley One & 585.976 & 484.611 & 0.00067 & 582.114 & 335.513 & 0.00065 & 583.12 & 304.542 & 0.00845 \\ 
                    \hline
                    Ackley Two & 582.764 & 549.495 & 0.00288 & 583.354 & 468.481 & 0.00286 & 583.037 & 462.064 & 0.03711 \\
                    \hline
                    De Jong 1 & 94620.3 & 64505.6 & 0.00003 & 99822.7 & 39754.1 & 0.00003 & 100140 & 29912.4 & 0.00042\\
                    \hline
                    Egg Holder & 94.5769 & -4812.16 & 0.00076 & -103.618 & -6552.22 & 0.00076 & -106.642 & -8759.94 & 0.00992 \\
                    \hline
                    Griewangk & 662.552 & 504.55 & 0.0009 & 625.738 & 219.673 & 0.00065 & 625.04 & 199.357 & 0.00793 \\
                    \hline
                    Rastrigin & 2.63e+06 & 1.79e+06 & 0.00029 & 2.7e+06 & 894380 & 0.00029 & 2.69e+06 & 902980 & 0.00374 \\
                    \hline
                    Rosenbrock's & 5.98e+10 & 3.18e+10 & 0.00127 & 5.79e+10 & 1e+10 & 0.00222 & 5.79e+10 & 9.46e+09 & 0.02876\\
                    \hline
                    Schwefel & 12367.4 & 9750.19 & 0.00039 & 12563.7 & 8469.67 & 0.00036 & 12556.5 & 6795.11 & 0.00472 \\
                    \hline
                    SESW & -21.3494 & -23.4479 & 0.00253 & -21.1728 & -28.3343 & 0.00253 & -21.1838 & -28.2668 & 0.0329 \\
                    \hline
                    SVSW & 301685 & 269094 & 0.0035 & 302376 & 227596 & 0.00347 & 303499 & 225942 & 0.04492 \\
                    \hline
                \end{tabular}
                \caption{Comparing 30, 50000, and 1000000 Tests in CUDA using XORWow}
            \end{centering}
            \end{table}

        \subsection{Empirical Gradient Descent}

            \begin{table}[h]
                \begin{centering}
                    \begin{tabular}{|c|c|c|c|c|c|c|c|c|c|c|}
                        \hline
                                        & Ackley 1 & Ackley 2 & De Jong 1 & Egg Holder & Griewangk \\
                        \hline
                        \hline
                        Minimum Xoshiro & 376.18  & 602.74   & 0.00073   & -10414.71  &           \\
                        \hline
                        Minimum Mersenne& 358.47  & 607.49   & 0.00075   & -12307.96  &           \\
                        \hline
                    \end{tabular}
                    \caption{Local optima with empirical gradient descent in Julia}
            \end{centering}
            \end{table}

            \begin{table}[h]
                \begin{centering}
                    \begin{tabular}{|c|c|c|c|c|c|c|c|c|c|c|}
                        \hline
                                        & Rastrigin & Rosenbrock            & Schwefel & SESW   & SVSW \\
                        \hline
                        \hline
                        Minimum Xoshiro & 358286.94 & $4.06 \times 10^{10}$ & 4125.82  & -29.73 &      \\
                        \hline
                        Minimum Mersenne& 148870.89 & $2.32 \times 10^{10}$ & 3869.72  & -29.36 &      \\
                        \hline
                    \end{tabular}
                    \caption{Local optima with empirical gradient descent in Julia}
            \end{centering}
            \end{table}
\pagebreak
    \section{Discussion}
        Compare empirical gradient to other standard methods in numerical optimization of geometric structures:
        \begin{itemize}
            \item Barzilai-Borwein method
            \item Wolfe conditions and Line search
            \item Conjugate gradient method
            \item Proximal gradient method
            \item Stochastic gradient descent
            \item Mirror descent
            \item Broyden–Fletcher–Goldfarb–Shanno algorithm
            \item Davidon–Fletcher–Powell formula
            \item Nelder–Mead method
            \item Gauss–Newton algorithm
            \item Quantum annealing
            \item Hill climbing
        \end{itemize}
\pagebreak
    \section{Conclusion}
\pagebreak
\nocite*{}
\bibliography{references}


\end{document}
